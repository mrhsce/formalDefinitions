\documentclass{article}

\usepackage[mathrm,colour,cntbysection]{czt}

% count definitions per section
\CountDefstrue

\begin{document}

\begin{center}
\begin{Huge}
Formal specification \\

{\huge Second assignment\\}
\end{Huge}
\begin{large}
\begin{flushleft}
Muhammad Reza Hidarian \t7 95210551
\\Esmaeel Gholami \t10 96210145
\end{flushleft}	
\end{large}\end{center}

\section{Finding minimum spanning tree for graph}

Based on the definition of the second project, our aim here is to check whether or not a specific tree is minimum spanning tree for a specific graph. The main function is 

\subsection*{Generic types}
The only generic type here is vertex.
\begin{zed}
  [Vertex]
\end{zed}

\subsection*{Weighted interlocked graph}
A weighted graph is a graph whose edges have a specific number assigned to them as weight and interlocked means that for every two different vertices there is a route connecting them. The following schema shows the identifiers and constraints for a weighted interlocked graph.
\begin{schema}{WI\_Graph}
  v : \finset Vertex\\
  e : \finset Vertex \fun \num
\where
  \forall x : \finset Vertex | x \in \dom e @ \# x = 2 \land x \subseteq v\\
  \forall v1, v2 : Vertex | v1 \in v \land v2 \in v @ \exists s : \seq Vertex @\\ \t1
   \forall i : \nat_1 | i < \# s @ \{s.1, s.i+1\} \in \dom e \land
  \{v1, s.1\} \in \dom e \land \\ \t2
  \{s.\#s, v2\} \in \dom e
\end{schema}

\subsection*{Weighted interlocked tree}
A weighted tree is an interlocked graph whose edges have a specific number assigned to them as weight and the count of vertices is one more than the count of edges. The following schema shows the identifiers and constraints for a weighted interlocked tree.
\begin{schema}{WI\_Tree}
  v : \finset Vertex\\
  e : \finset Vertex \fun \num
\where
  \forall x : \finset Vertex | x \in \dom e @ \# x = 2 \land x \subseteq v\\
  \forall v1, v2 : Vertex | v1 \in v \land v2 \in v @ \exists s : \seq Vertex @\\ \t1
   \forall i : \nat_1 | i < \# s @ \{s.1, s.i+1\} \in \dom e \land
  \{v1, s.1\} \in \dom e \land \\ \t2
  \{s.\#s, v2\} \in \dom e\\
  \#v = \#\dom e + 1
\end{schema}

\subsection*{Calculate weight function}
The following function takes the edges of a graph and gives the summation of weights of all edges.
\begin{axdef}
  weight : (\finset vertex \fun \num)\fun \num
\where
  \forall x : \finset vertex \fun \num @ \\
   \#x = 0 \fun weight x = 0 \\ \#x \neq 0 \fun \exists e \in x @ weight~x = e.2 + weight~(x-e)
\end{axdef}

\subsection*{check is spanning tree function}
The following function takes a tree and a graph and checks whether or not the tree is the spanning tree of the graph.
\begin{axdef}
  spanningTree~\_~\_ : WI\_Tree \cross WI\_Graph
\where
  \forall wt : WI\_Tree, wg : WI\_Graph @ \\
  spanningTree~wt~wg \iff wg.v = wt.v
\end{axdef}

\subsection*{check is minimum spanning tree function}
This is the main function and takes a tree and a graph and checks whether or not the tree is the minimum spanning tree of the graph. 1 meas positive and 0 means negative.
\begin{axdef}
  minimumSpanningTree~\_~\_ : WI\_Tree \cross WI\_Graph
\where
  \forall wt : WI\_Tree, wg : WI\_Graph @ \\
  spanningTree~wt~wg \iff \\ \forall wt1 : WI\_Tree | spanning~wt1~wg @ weight~wt \leq weight~wt1
\end{axdef}

\section{Musical chair formal definition}
Based on the description of the game there are two generic types:
\begin{zed}
  [Player, Chair]
\end{zed}

The schema of the game is as follows:

\begin{schema}{MusicalChair}
  pl : \finset Player\\
  ch : \finset Chair \\
  occupiers : Chair  \pinj Player
\where
  \# ch \geq 1\\
  \# pl = \# ch +1\\
  \dom occupiers \subseteq ch\\
  \ran occupiers \subseteq pl
\end{schema}

\subsection*{MusicPlay}
The operation for when the music starts and keeps playing:
\begin{schema}{MusicPlay}
  \Delta MusicalChair
\where
  pl' = pl\\
  ch' = ch\\
  occupiers' = \emptyset
\end{schema}

\subsection*{MusicStop}
The operation for when the music stops:
\begin{schema}{MusicStop}
  \Delta MusicalChair\\
  out! : Player
\where
  ch' = ch\\
  \dom occupiers' = ch'\\
  out! \in pl - \ran occupiers'
\end{schema}

\subsection*{RemoveLoser}
The operation for when the loser should be removed:
\begin{schema}{RemoveLoser}
  \Delta MusicalChair\\
  out? : Player
\where
  pl' = pl - \{ out? \}\\
  \exists chair : Chair | chair \in ch @ ch' = ch - \{chair\}
\end{schema}

\subsection*{Game}
The game itself:
\begin{zed}
MusicalChairInit == MusicalChairs'; p? : \finset Player; c?:\finset Chair | \\ \t1
\#c? \geq 1 \land \# p? = \# c? + 1 \land pl' = p? \land ch' = c?  \\\\
  
  GameFinished == MusicalChairs; winner! : Player | \\ \t1
  \# occupiers = 1 \land winner! \in \ran occupiers \\\\
  
  PlayMusicalChairs == MusicalChairInit \typecolon \\ \t1
   (MusicPlay \typecolon MusicStop \typecolon RemoveLoser[out!/out?])^{n} \typecolon \\ \t1 GameFinished \\ \t2
    Where~n= \# c? -1
\end{zed}

\end{document}